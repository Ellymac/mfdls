\documentclass[12pt]{article}

\usepackage[utf8]{inputenc}
\usepackage{xcolor}

\pagestyle{plain}
\title{Projet MFDLS \\ Vélos en libre service}
\author{Anthony Araye et Camille Schnell}
\date{1 mars 2018}
\begin{document}
\maketitle
\newpage
\renewcommand{\contentsname}{Sommaire}
\tableofcontents
\newpage
\section{Introduction}
L'objectif de ce projet est de modéliser un système permettant à des utilisateurs d'emprunter des vélos en libre service. Ces vélos sont stockés dans plusieurs gares au sein d'une ville. \\ \\
Il s'agit ici de gérer l'ensemble des actions pouvant être effectuées par différentes personnes : des usagers ou des administrateurs du système. Nous avons ainsi modélisé des opérations afin qu'un administrateur puisse créer, supprimer ou débloquer un utilisateur et déplacer ou réparer des vélos ; les usagers pourront emprunter et rendre des vélos, à condition qu'ils ne soient pas bloqués. \\ \\
Les raffinements permettent de ...
\newpage
\section{La machine abstraite}
\textcolor{red}{\textbf{IL FAUT PRÉCISER DANS CETTE SECTION À CHAQUE FOIS QU'UNE PROPRIÉTÉ DU SUJET EST VÉRIFIÉE EN \\ LA CITANT (P1,P2,...)}}
\subsection{Ensembles, variables et initialisation}
\subsection{Opérations implémentées}
\subsubsection{\textit{creer\_usager(user)}}
\subsubsection{\textit{supprimer\_usager(user)}}
\subsubsection{\textit{debloquer\_usager(user)}}
\subsubsection{\textit{emprunter(user,vv,gg)}}
\subsubsection{\textit{rendre(velo,gg,temps,etat)}}
\subsubsection{\textit{deplacer\_velo(vv,gg)}}
\subsubsection{\textit{reparer\_velo(velo)}}
\newpage
\section{Scénarii de test}
Voici quelques scénarii utilisés pour tester la machine abstraite Velhop. Après initialisation, le résultat obtenu est le suivant : \\
invariant\_ok \\
velos = \{1,2,3,4,5,6,7,8,9,10,11,12\};
gare = \{1,2,3,4\} \\
etatVelos(1) = bon\_etat
etatVelos(2) = bon\_etat
etatVelos(3) = bon\_etat
etatVelos(4) = bon\_etat
etatVelos(5) = bon\_etat
etatVelos(6) = bon\_etat
etatVelos(7) = bon\_etat
etatVelos(8) = bon\_etat
etatVelos(9) = bon\_etat
etatVelos(10) = bon\_etat
etatVelos(11) = bon\_etat
etatVelos(12) = bon\_etat \\
usagers = \{\};
usagerVelo = \{\} \\
gares(1) = [1,2,3]
gares(2) = [4,5,6]
gares(3) = [7,8,9]
gares(4) = [10,11,12]
\subsection{Premier scénario}
\begin{itemize}
  \item Initialisation
  \item Création d'un utilisateur 1 \\
  $\Rightarrow$ \textit{usagers(PERSONNES1) = actif}
  \item L'utilisateur 1 emprunte le vélo 1 à la gare 1 \\
  $\Rightarrow$ \textit{usagers(PERSONNES1) = actif ; usagerVelo(PERSONNES1)=1 ; $gares(1)$ = \{$(2\mapsto2)$,$(3\mapsto3)$\}}
  \item L'utilisateur 1 rend le vélo 1 en bon état à la gare 1 après 30 minutes \\
  $\Rightarrow$ \textit{usagers(PERSONNES1) = actif ; usagerVelo=\{\} ; gares(1) = [1,2,3]}
  \item L'utilisateur 1 est supprimé \\
  $\Rightarrow$ \textit{usagers = \{\}}
\end{itemize}
\subsection{Deuxième scénario}
\begin{itemize}
  \item Initialisation
  \item Création d'un utilisateur 1 \\
  $\Rightarrow$ \textit{usagers(PERSONNES1) = actif}
  \item L'utilisateur 1 emprunte le vélo 4 à la gare 2 \\
  \item L'utilisateur 1 rend le vélo 4 abimé à la gare 3 après 45 minutes : il est bloqué
  $\Rightarrow$ \textit{}
  \item Le vélo 4 est réparé
  \item L'utilisateur 1 est débloqué
  \item L'utilisateur 1 emprunte le vélo 4 à la gare 3
  \item L'utilisateur 1 rend le vélo 4 en bon état à la gare 1 après 30 minutes
  \item L'utilisateur 1 est supprimé
\end{itemize}
\subsection{Troisième scénario}
\begin{itemize}
  \item Initialisation
  \item Création d'un utilisateur 1
  \item Création d'un utilisateur 2
  \item L'utilisateur 1 emprunte le vélo 4 à la gare 2
  \item L'utilisateur 1 rend le vélo 4 en bon état à la gare 3 après 70 minutes : il est bloqué
  \item L'administrateur déplace le vélo 4 dans la gare 1
  \item L'utilisateur 2 emprunte le vélo 4 à la gare 1
  \item L'utilisateur 2 rend le vélo 4 en bon état à la gare 1 après 30 minutes
  \item L'utilisateur 1 est supprimé
  \item L'utilisateur 2 est supprimé
\end{itemize}
\newpage
\section{Premier raffinement}
\subsection{Invariant de collage}
\subsection{Modification des opérations}
\newpage
\section{Deuxième raffinement}
\subsection{Invariant de collage}
\subsection{Modification des opérations}
\section{Preuves}
\subsection{Preuves de la machine abstraite}
\subsection{Preuves du premier raffinement}
\subsection{Preuves du deuxième raffinement}
\newpage
\section{Conclusion}



\end{document}
